\author{Tavo Annus, Timo Loomets, Mattias Kitsing}
\title{%
   Overview of mapping algorithms \\
  \large Team SCITOS-02}
\date{\today}

%%%%%%%%%%%%%%%%%
% Configuration %
%%%%%%%%%%%%%%%%%

\documentclass[12pt, a4paper, onecolumn]{article}
\usepackage{xurl}
\usepackage[super,comma,sort&compress]{natbib}
\usepackage{graphicx}
\usepackage{abstract}
\usepackage{enumitem}
\usepackage{lipsum,booktabs}
\renewcommand{\abstractnamefont}{\normalfont\bfseries}
\renewcommand{\abstracttextfont}{\normalfont\small\itshape}
\usepackage{lipsum}

% Any configuration that should be done before the end of the preamble:
\usepackage{hyperref}
\hypersetup{colorlinks=true, urlcolor=blue, linkcolor=black, citecolor=blue}
\setlength{\belowcaptionskip}{-10pt}
\pagenumbering{gobble}
\renewcommand\thesection{}
\setlength{\parindent}{0em}
\setlength{\parskip}{0,5em}
\addtolength{\topmargin}{-60pt}
\addtolength{\textheight}{120pt}
\linespread{1.25}
\pagenumbering{arabic}

\begin{document}
\maketitle
%%%%%%%%%%%
% Article %
%%%%%%%%%%%

% Intro %
\section{Introduction}
World models are representations of reality which help robots to localize within and interact with their environment. For successful utilization of the models they have to be:
\begin{itemize}
  \item Compact to be used efficiently
  \item Adapted to the task and the environment
  \item Usable in case of sensor and position uncertainty
  \end{itemize}
A map is a commonly used model of the environment and very essential in mobile robotics. Maps allow robots to localize themselves and carry out their tasks. A variety of maps can be employed and we can distinguish them by the metrics they use (continuous, discrete, topological).\par
The scope of this paper is to analyze different mapping for mobile robots with special focus on 2D motion. The task was to write about different maps and mapping algorithms including advantages and disadvantages for each. With that overview the team shall choose which map representation makes most sense for to implement during the mapping assignments. \par
Note: \emph{the focus should be on mobile robots moving in a 2D space using information from a range-finder.}

% 1st algorithm %
\section{Cartographer Algorithm}
Cartographer is Google's real-time indoor mapping project.
The sensor is mounted on a backpack and can generate a 2D
grid map with a resolution of 5 cm. Scan match is used to insert
each frame of scan data obtained by the laser radar into a
submap at the best estimated position, and scan matching is
only related to the current submap.
\begin{table}[h!]
\centering
\begin{tabular}{ |p{6cm}||p{6cm}|  }
 \hline
 Advantages&Disadvantages\\
 \hline
\begin{itemize}
  \item Robust
  \item Sample Text
  \end{itemize} 
  & 
  \begin{itemize}
  \item Sample Text
  \item Sample Text
  \end{itemize} \\

 \hline
\end{tabular}
\caption{1st algorithm pro's and cons }
\label{table:1}
\end{table}

% 2nd algorithm %
\section{GMapping Algorithm}

% 3rd algorithm %
\section{3rd algorithm?}

% Conclusion %
\section{Conclusion}

% Summary %
\section{Summary}
\newpage
\begin{thebibliography}{9}
\bibitem{texbook}
Donald E. Knuth (1986) \emph{The \TeX{} Book}, Addison-Wesley Professional.

\bibitem{lamport94}
Leslie Lamport (1994) \emph{\LaTeX: a document preparation system}, Addison
Wesley, Massachusetts, 2nd ed.
\end{thebibliography}

\end{document}

