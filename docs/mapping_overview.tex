\author{Tavo Annus, Timo Loomets, Mattias Kitsing}
\title{%
   Overview of mapping algorithms \\
  \large Team SCITOS-02}
\date{\today}

%%%%%%%%%%%%%%%%%
% Configuration %
%%%%%%%%%%%%%%%%%

\documentclass[12pt, a4paper, onecolumn]{article}
\usepackage{xurl}
\usepackage[super,comma,sort&compress]{natbib}
\usepackage{graphicx}
\usepackage{abstract}
\usepackage{enumitem}
\usepackage{lipsum,booktabs}
\renewcommand{\abstractnamefont}{\normalfont\bfseries}
\renewcommand{\abstracttextfont}{\normalfont\small\itshape}
\usepackage{lipsum}

% Any configuration that should be done before the end of the preamble:
\usepackage{hyperref}
\hypersetup{colorlinks=true, urlcolor=blue, linkcolor=black, citecolor=blue}
\setlength{\belowcaptionskip}{-10pt}
\pagenumbering{gobble}
%\pagenumbering{arabic}
\renewcommand\thesection{}
\setlength{\parindent}{0em}
\setlength{\parskip}{0,8em}
\addtolength{\topmargin}{-60pt}
\addtolength{\textheight}{120pt}
\linespread{1.25}

\begin{document}
\maketitle
%%%%%%%%%%%
% Article %
%%%%%%%%%%%

% Intro %
\section{Introduction}
World models are representations of reality which help robots to localize within and interact with their environment. For successful utilization of the models they have to be:
\begin{itemize}
  \item Compact to be used efficiently
  \item Adapted to the task and the environment
  \item Usable in case of sensor and position uncertainty
  \end{itemize}
A map is a commonly used model of the environment and very essential in mobile robotics. Maps allow robots to localize themselves and carry out their tasks. A variety of maps can be employed and we can distinguish them by the metrics they use (continuous, discrete, topological).\par
The scope of this paper is to analyze different mapping for mobile robots with special focus on 2D motion. The task was to write about different maps and mapping algorithms including advantages and disadvantages for each. With that overview the team shall choose which map representation makes most sense for to implement during the mapping assignments. \par
Note: \emph{the focus should be on mobile robots moving in a 2D space using information from a range-finder.}

% 1st algorithm %
\section{Cartographer Algorithm}
When the amount of data to process becomes too large, particle-based algorithms are not applicable due to their higher computing requirements on the processor. In this case, graph optimisation algorithms are more suitable.\cite{2DSLAM20}\par
Cartographer is Google's real-time indoor mapping project. The Google open source code consists of two parts: Cartographer and Cartographer{\_}ROS. The function of Cartographer is to process the data from Lidar, IMU, and odometers to build a map. Cartographer{\_}ROS then acquires the sensor data through the ROS communication mechanism and converts them into the Cartographer format for processing by Cartographer, while the Cartographer processing result is released for display or storage. \cite{2DSLAM20} \par Cartographer uses global map optimization cycle and local probabilistic map updates, it makes this system more robust to environmental changes. 
Impressive real-time results for solving SLAM in 2D have been described in \cite{CARTO16} by the authors of the software.
\begin{table}[h!]
\centering
\begin{tabular}{ |p{6,5cm}||p{6,5cm}|  }
 \hline
 Advantages&Disadvantages\\
 \hline
\begin{itemize}
  \item Robust (low dispersion) \cite{SLAMQuality}
  \item Good documentation
  \item Feasible for large amount of data
  \item Accurate \cite{SLAMQuality}
  \end{itemize} 
  & 
  \begin{itemize}
  \item May fail on some sequences (create inconsistent map)\cite{SLAMQuality}
  \item Possibility of blurry effect \cite{SLAMQuality}
  \item Complex
  \end{itemize} \\

 \hline
\end{tabular}
\caption{Cartographer algorithm pro's and cons }
\label{table:1}
\end{table}

Possible optimization for Cartographer and its parameters can be found here: \cite{CARTO19}
% 2nd algorithm %
\section{GMapping Algorithm}
GMapping was developed in 2007, and it is still the one of the most common systems for robot application.The system uses Particle filter and creates the grid-based map. Gmapping is one of the most robust algorithms. This algorithm is good in handling noisy data and short laser measurements in difficult environments \cite{SLAMbench17}
\begin{table}[h!]
\centering
\begin{tabular}{ |p{6cm}||p{6cm}|  }
 \hline
 Advantages&Disadvantages\\
 \hline
\begin{itemize}
  \item Robust (low dispersion) \cite{SLAMQuality}
  \item No blurry effect \cite{SLAMQuality}
  \item Accurate and fast for small maps \cite{SLAMQuality20}
  \end{itemize} 
  & 
  \begin{itemize}
  \item Depends heavily on odometry data
  \item No loop detection \cite{SLAMQuality20}
  \item Not suitable for building a large scene map \cite{SLAMQuality20}
  \end{itemize} \\

 \hline
\end{tabular}
\caption{GMapping algorithm pro's and cons}
\label{table:2}
\end{table}

Improvement of GMapping and its parameters can be found here: \cite{SLAMQuality20}

% 3rd algorithm %
\section{3rd algorithm?}

% Conclusion %
\section{Conclusion}

% Summary %
\section{Summary}
\newpage
\begin{thebibliography}{9}
\bibitem{2DSLAM20}
Xuexi Zhang, Jiajun Lai, Dongliang Xu, Huaijun Li, Minyue Fu, (2020) "2D Lidar-Based SLAM and Path Planning for Indoor Rescue Using Mobile Robots", Journal of Advanced Transportation, vol. 2020, Article ID 8867937, 14 pages, 2020. https://doi.org/10.1155/2020/8867937

\bibitem{SLAMQuality}
Filatov, Anton and Filatov, Artyom and Krinkin, Kirill and Chen, Baian and Molodan, Diana. (2017). "2D SLAM Quality Evaluation Methods". 

\bibitem{CARTO19}
Zhi, Cui, (2019), "Research on Cartographer Algorithm based on Low Cost Lidar". International Journal of Engineering Research and. V8. 10.17577/IJERTV8IS100060. 
\bibitem{CARTO16}
W. Hess, D. Kohler, H. Rapp and D. Andor, "Real-time loop closure in 2D LIDAR SLAM," 2016 IEEE International Conference on Robotics and Automation (ICRA), 2016, pp. 1271-1278, doi: 10.1109/ICRA.2016.7487258.
\bibitem{SLAMbench17}
Peter Aerts, Eric Demeester (2017) "Benchmarking of 2D-Slam Algorithms
A Validation for the TETRA project Ad Usum Navigantium", ACRO, http://www.acro.be/downloadvrij/Benchmark{\_}2D{\_}SLAM.pdf
\bibitem{SLAMQuality20}
Z. Meng, C. Wang, Z. Han and Z. Ma, "Research on SLAM navigation of wheeled mobile robot based on ROS," 2020 5th International Conference on Automation, Control and Robotics Engineering (CACRE), 2020, pp. 110-116, doi: 10.1109/CACRE50138.2020.9230186.
\end{thebibliography}

\end{document}

