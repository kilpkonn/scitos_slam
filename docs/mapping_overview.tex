\author{Tavo Annus, Timo Loomets, Mattias Kitsing}


%%%%%%%%%%%%%%%%%
% Configuration %
%%%%%%%%%%%%%%%%%

\documentclass[12pt, a4paper, twocolumn]{article}
\usepackage{xurl}
\usepackage[super,comma,sort&compress]{natbib}
\usepackage{graphicx}
\usepackage{abstract}
\usepackage{enumitem}
\renewcommand{\abstractnamefont}{\normalfont\bfseries}
\renewcommand{\abstracttextfont}{\normalfont\small\itshape}
\usepackage{lipsum}

% Any configuration that should be done before the end of the preamble:
\usepackage{hyperref}
\hypersetup{colorlinks=true, urlcolor=blue, linkcolor=black, citecolor=blue}
\setlength{\belowcaptionskip}{-10pt}
\pagenumbering{gobble}
\renewcommand\thesection{}
\setlength{\parindent}{0em}
\setlength{\parskip}{0,5em}
\addtolength{\topmargin}{-60pt}
\addtolength{\textheight}{120pt}
\linespread{1.25}
\begin{document}

%%%%%%%%%%%
% Article %
%%%%%%%%%%%
\textbf{Team: SCITOS-02}

\section{Introduction}
A map is a commonly used model of the environment and very essential in mobile robotics. Maps allow robots to localize themselves and carry out their tasks. A variety of maps can be employed and we can distinguish them by the metrics they use ( continuous, discrete, topological).\par
The scope of this paper is to analyze different mapping for mobile robots with special focus on 2D motion. The task was to write about different maps and mapping algorithms including advantages and disadvantages for each. With that overview the team shall choose which map representation makes most sense for to implement during the mapping assignments. \par
Note: \emph{ the focus should be on mobile robots moving in a 2D space using information from a range-finder.}

\section{1st algorithm}
\par 
\end{document}

