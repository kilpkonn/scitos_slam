\author{Tavo Annus, Timo Loomets, Mattias Kitsing}
\title{%
   Overview of mapping algorithms \\
  \large Team SCITOS-02}
\date{\today}

%%%%%%%%%%%%%%%%%
% Configuration %
%%%%%%%%%%%%%%%%%

\documentclass[12pt, a4paper, onecolumn]{article}
\usepackage{xurl}
\usepackage[super,comma,sort&compress]{natbib}
\usepackage{graphicx}
\usepackage{abstract}
\usepackage{enumitem}
\usepackage{lipsum,booktabs}
\renewcommand{\abstractnamefont}{\normalfont\bfseries}
\renewcommand{\abstracttextfont}{\normalfont\small\itshape}
\usepackage{lipsum}

% Any configuration that should be done before the end of the preamble:
\usepackage{hyperref}
\hypersetup{colorlinks=true, urlcolor=blue, linkcolor=black, citecolor=blue}
\setlength{\belowcaptionskip}{-10pt}
\pagenumbering{gobble}
%\pagenumbering{arabic}
\renewcommand\thesection{}
\setlength{\parindent}{0em}
\setlength{\parskip}{0,8em}
\addtolength{\topmargin}{-60pt}
\addtolength{\textheight}{120pt}
\linespread{1.25}

\begin{document}
\maketitle
%%%%%%%%%%%
% Article %
%%%%%%%%%%%

% Intro %
\section{Introduction}
A map is a commonly used model of the environment and very essential in mobile robotics. Maps allow robots to localize themselves and carry out their tasks.
The scope of this paper is to analyze different mapping for mobile robots with special focus on 2D motion.
The main focus points are on the compactness (memory usage) of the map, robustness and adoption to indoors space.


\section{Data models for maps}
The most basic way to store the map is memory is to store all the measurement datapoints (after removing noise etc.).
However this causes heavy usage of memory that makes it not efficient.
The easieiest way to decrease the number of datapoints is to use an occupancy grid for map where each tile is for example a square or hexagon for 2D use case.
One problem with this kind of approach is that the grid size affects the quality of the map a lot.
With big tiles we lose precision whilst with small tiles we might not have enough datapoints for each time which can cause errors in map.
To keep the precision where we have enough datapoints but not suffer from underfitting instead of tile grid trees can be used.
The motivation is that we can start with very robust map and once any cell gets enough datapoints we split it into smaller points.
The thirs option is to try to detect geometrical objects (for example lines) and store these.
These lines can be obtained by either linear regression or iterative point fitting algorithms.
This results in a much smaller memory consumption and can give also more insight into the actual terrain in urban areas, wheras can yield poor results in more natural environment.
\cite{AlgorithmsForIndoorMapping}
The comparison of the algorithms described above can be found in the Table~\ref{tab:memory_models}.

\begin{table}[h!]
  \begin{center}
    \begin{tabular}[c]{|c | c c c|}
      \hline
       & Raw points & Occupancy grid & Geometric features \\
      \hline
      Impl. difficulty & Easy & Easy - Medium & Hard \\
      \hline
      Memory usage & Huge & Medium & Small \\
      \hline
      CPU usage* & Small & Medium & Big \\
      \hline
      Robustness & Good & Good & Environment specific \\
      \hline
      Continuous & Yes & No & Yes \\
      \hline
    \end{tabular}
  \end{center}
  \caption{Comparison of different memory models}
  \label{tab:memory_models}
\end{table}
\textit{*This accounts for CPU consumption when collecting data, not when later using it.}

\section{Graph SLAM}
Graph SLAM is an algorithm that uses probabilistic approach to simultaneously locate iteself and map new environment.
Each instead datapoint (current location, location of a wall, etc.) is represented by a Gaussian distribution that peaks at where the measurement was
and has standard deviation inversely propotional to confidence in our measurement.
All these datapoints are then viewed as constraints for the system that needs to be maximized.
The locations for the robot and objects that maximize the system are then used as they are the most probable ones.
For this algorithm to make sense we need to have multiple estimates (essentially constraints) for every point of interest so using raw datapoints is not acceptable.
Both occupancy grid and gemoetrical representation of the environment can work well, however it may be possible to decrease the CPU usage with using geometrical lines (if there are fewer than grid tiles).

\section{EKF SLAM}
It is also possible to use Extended Kalman Filter for SLAM as current pose and also points of interest locations can be built from previous estimates and state transitions.
The EKF SLAM algorithm scales better than Graph SLAM in the states dimension \cite{EKFSLAM}. 
However the model is a lot more sensitive to incorrect associations (points of interest get mixed).
The algorithm is also seems to be harder to implement than graph SLAM (?). 

\section{Enhanced adaptive fuzzy clustering (EAFC) with noise clustering SLAM \cite{AlgorithmsForIndoorMapping}}
This approach focuses on reducing noise before creating a map because noise can lead to an incorrect map.
Many of the methods used are inspired by image processing such as mathemathical erosion and dilation.
Since this approach uses fuzzy c-means clustering which is an extension of k-means clustering then this would also be harder to implement.
This approach also relies on tuning the filtering and mapping parameters to the noise that is present which makes it less robust.

\section{Particle filter SLAM \cite{ParticleFilterSLAM}}
The particle filter SLAM-s main benefit is the low bar of entry for implementation.
Another great benefit is it's robustness given enough compute.
Difficulties may rise from defining accurate movement model of the platform.
The increase in noise and estimation errors increases the requirements for more particles where each particle requires significant memory and computational time.
The precision of this method is easy to evaluate on the go by the grouping of particles, but in difficult situations precision will start to diminish and localisation can start to jump.
A benefit of this method is also it's easy integration with different types of maps.


% % 1st algorithm %
% \section{Cartographer Algorithm}
% When the amount of data to process becomes too large, particle-based algorithms are not applicable due to their higher computing requirements on the processor. In this case, graph optimisation algorithms are more suitable.\cite{2DSLAM20}\par
% Cartographer is Google's real-time indoor mapping project. The Google open source code consists of two parts: Cartographer and Cartographer{\_}ROS. The function of Cartographer is to process the data from Lidar, IMU, and odometers to build a map. Cartographer{\_}ROS then acquires the sensor data through the ROS communication mechanism and converts them into the Cartographer format for processing by Cartographer, while the Cartographer processing result is released for display or storage. \cite{2DSLAM20} \par Cartographer uses global map optimization cycle and local probabilistic map updates, it makes this system more robust to environmental changes. 
% Impressive real-time results for solving SLAM in 2D have been described in \cite{CARTO16} by the authors of the software.
% \begin{table}[h!]
% \centering
% \begin{tabular}{ |p{6,5cm}||p{6,5cm}|  }
%  \hline
%  Advantages&Disadvantages\\
%  \hline
% \begin{itemize}
%   \item Robust (low dispersion) \cite{SLAMQuality}
%   \item Good documentation
%   \item Feasible for large amount of data
%   \item Accurate \cite{SLAMQuality}
%   \end{itemize} 
%   & 
%   \begin{itemize}
%   \item May fail on some sequences (create inconsistent map)\cite{SLAMQuality}
%   \item Possibility of blurry effect \cite{SLAMQuality}
%   \item Complex
%   \end{itemize} \\
%
%  \hline
% \end{tabular}
% \caption{Cartographer algorithm pro's and cons }
% \label{table:1}
% \end{table}
%
% Possible optimization for Cartographer and its parameters can be found here: \cite{CARTO19}
% % 2nd algorithm %
% \section{GMapping Algorithm}
% GMapping was developed in 2007, and it is still the one of the most common systems for robot application.The system uses Particle filter and creates the grid-based map. Gmapping is one of the most robust algorithms. This algorithm is good in handling noisy data and short laser measurements in difficult environments \cite{SLAMbench17}
% \begin{table}[h!]
% \centering
% \begin{tabular}{ |p{6cm}||p{6cm}|  }
%  \hline
%  Advantages&Disadvantages\\
%  \hline
% \begin{itemize}
%   \item Robust (low dispersion) \cite{SLAMQuality}
%   \item No blurry effect \cite{SLAMQuality}
%   \item Accurate and fast for small maps \cite{SLAMQuality20}
%   \end{itemize} 
%   & 
%   \begin{itemize}
%   \item Depends heavily on odometry data
%   \item No loop detection \cite{SLAMQuality20}
%   \item Not suitable for building a large scene map \cite{SLAMQuality20}
%   \end{itemize} \\
%
%  \hline
% \end{tabular}
% \caption{GMapping algorithm pro's and cons}
% \label{table:2}
% \end{table}
%
% Improvement of GMapping and its parameters can be found here: \cite{SLAMQuality20}

% Conclusion %
\section{Conclusion}

\newpage
\begin{thebibliography}{9}
\bibitem{2DSLAM20}
Xuexi Zhang, Jiajun Lai, Dongliang Xu, Huaijun Li, Minyue Fu, (2020) "2D Lidar-Based SLAM and Path Planning for Indoor Rescue Using Mobile Robots", Journal of Advanced Transportation, vol. 2020, Article ID 8867937, 14 pages, 2020. https://doi.org/10.1155/2020/8867937

\bibitem{SLAMQuality}
Filatov, Anton and Filatov, Artyom and Krinkin, Kirill and Chen, Baian and Molodan, Diana. (2017). "2D SLAM Quality Evaluation Methods". 

\bibitem{CARTO19}
Zhi, Cui, (2019), "Research on Cartographer Algorithm based on Low Cost Lidar". International Journal of Engineering Research and. V8. 10.17577/IJERTV8IS100060. 
\bibitem{CARTO16}
W. Hess, D. Kohler, H. Rapp and D. Andor, "Real-time loop closure in 2D LIDAR SLAM," 2016 IEEE International Conference on Robotics and Automation (ICRA), 2016, pp. 1271-1278, doi: 10.1109/ICRA.2016.7487258.
\bibitem{SLAMbench17}
Peter Aerts, Eric Demeester (2017) "Benchmarking of 2D-Slam Algorithms
A Validation for the TETRA project Ad Usum Navigantium", ACRO, http://www.acro.be/downloadvrij/Benchmark{\_}2D{\_}SLAM.pdf
\bibitem{SLAMQuality20}
Z. Meng, C. Wang, Z. Han and Z. Ma, "Research on SLAM navigation of wheeled mobile robot based on ROS," 2020 5th International Conference on Automation, Control and Robotics Engineering (CACRE), 2020, pp. 110-116, doi: 10.1109/CACRE50138.2020.9230186.

\bibitem{AlgorithmsForIndoorMapping}
  Ankit A. Ravankar, Yohei Hoshino, Abhijeet Ravankar, Lv Jixin, Takanori Emaru, Yukinori Kobayashi, (2015) "Algorithms and a Framework for Indoor Robot Mapping in a Noisy Environment Using Clustering in Spatial and Hough Domains", https://journals.sagepub.com/doi/full/10.5772/59992
\bibitem{EKFSLAM}
  McGill (2017) "EKF and SLAM" https://www.cim.mcgill.ca/~dmeger/comp765/slides/COMP765-Fall2017-Lecture5-Mapping.pdf
\bibitem{ParticleFilterSLAM}
  Norhidayah Mohamad Yatima, Norlida Buniyamin, (2015) "Particle Filter in Sinultaneous Lolaization and Mapping (SLAM) Using Differential Drive Mobile Robot", https://www.researchgate.net/publication/286763834{\_}Particle{\_}filter{\_}in{\_}simultaneous{\_}localization{\_}and{\_}mapping{\_}Slam{\_}using{\_}differential{\_}drive{\_}mobile{\_}robot
\end{thebibliography}

\end{document}

