\author{Tavo Annus, Timo Loomets, Mattias Kitsing}
\title{%
   Overview of path planning and obstacle avoidance algorithms \\
  \large Team SCITOS-02}
\date{\today}

%%%%%%%%%%%%%%%%%
% Configuration %
%%%%%%%%%%%%%%%%%

\documentclass[12pt, a4paper, onecolumn]{article}
\usepackage{xurl}
\usepackage[super,comma,sort&compress]{natbib}
\usepackage{graphicx}
\usepackage{abstract}
\usepackage{enumitem}
\usepackage{lipsum,booktabs}
\renewcommand{\abstractnamefont}{\normalfont\bfseries}
\renewcommand{\abstracttextfont}{\normalfont\small\itshape}
\usepackage{lipsum}

% Any configuration that should be done before the end of the preamble:
\usepackage{hyperref}
\hypersetup{colorlinks=true, urlcolor=blue, linkcolor=black, citecolor=blue}
\setlength{\belowcaptionskip}{-10pt}
\pagenumbering{gobble}
%\pagenumbering{arabic}
\renewcommand\thesection{}
\setlength{\parindent}{0em}
\setlength{\parskip}{0,8em}
\addtolength{\topmargin}{-60pt}
\addtolength{\textheight}{120pt}
\linespread{1.25}

\begin{document}
\maketitle
%%%%%%%%%%%
% Article %
%%%%%%%%%%%

% Intro %
\section{Introduction}

The third essential ingredient for robot autonomy deals with higher level decision making regarding the movement of the robot in its environment. 
Given a starting configuration (pose in our case), an end configuration, a representation of the environment surrounding the robot (assumed to be known for this assignment) and information about the environment from on-board sensors, the task is to define a feasible path through the environment to reach the end configuration from the start configuration without violating given constraints of the robot and its environment.
The environment may be static or dynamic and the planning may exhibit different layers of abstraction. 

Aside from the global planning task for the complete path, the robot needs to react to obstacles on a local level, which is especially important for dynamic environments.
This reaction is usually defined as obstacle or collision avoidance and usually employs a more accurate local planner.

\section{A* algorithm}

The A* Algorithm is a popular graph traversal path planning algorithm that improves on Dijsktra's algorithm with some heuristics.
It is widely used in static environments, but there are still examples where it is used in dynamic environments.
This algorithm works based on the lowest cost path tree from the initial point to the final target point. \cite{Karur21}

It is simpler and less computationally-heavy than many other path planning algorithms, with its efficiency lending itself to operation on constrained and embedded systems. It is
suitable for applications deployed in static environments. \cite{Karur21}

\section{D* algorithm}

D* algorithm or Dynamic A* algorithm is used to generate a collision-free path amidst moving obstacles.
D* is an informed incremental search algorithm that repairs the cost map partially and the previously calculated cost map. \cite{Karur21}
Opposed to A*, D* calculates path from goal to start to make recalculation of the path more efficient. \cite{DstarWiki}
It also recalculates only part of path that is required opposed to recalculating whole path as A* does.
D* star has also many derivatives out of which D* Lite seems the most appealing due to it's efficiency and it being easier to implement than most other D* derivatives.

\section{Global path algorithm for feature map}

In this \cite{Ren22} paper a global path planning algorithm based on the feature map is proposed using on the direction of line segment features. They state that simulation experiments demonstrate
that the proposed algorithm is superior to A* algorithm in terms of computation time
and path length, especially of the computation efficiency. \cite{Ren22}

The algorithm tries to draw a straight line from start to goal and upon facing obstacle the path is divided into two, one going left from there and another going right till the end of the obstacle.
From there it performs the same loop recursively to find all the paths to end goal.
After that optimization is done to cut out unnecessary corners. \cite{Ren22} % This cite should be for whole paragraph

\section{Optimized Rapidly-exploring Random Tree algorithm}

Optimized Rapidly-exploring random tree (RRT*) \cite{Karaman11} is a path planning algorithm that uses Sampling-based methods for optimal motion planning. 
The improvements from a regular RRT algorithm are that the algorithm records the distance each vertex has traveled relative to its parent vertex \cite{RRT19}.
After the closest node is found in the graph, a neighborhood of vertices in a fixed radius from the new node are examined \cite{RRT19}.
Also after a vertex has been connected to the cheapest neighbor, the neighbors are again examined \cite{RRT19}.
These improvements help to make straighter and very reliable paths .

RRT* also has a obstacle avoiding functionality, where this is checked against every time a node is placed, connected to a neighbour or a node is rewired.
These checks uses a lot of the computational resources. \cite{RRT19}

% Conclusion %
\section{Path Planning Comparison}
The comparison of different path planning algorithms are done in 
Table~\ref{tab:path_algorithms}.
\begin{table}[h!]
  \begin{center}
    \begin{tabular}[c]{|c || c c c c |}
      \hline
       Category        &  A*            & D*      & Global & RRT*    \\
      \hline
      Impl. difficulty & low-medium     & medium  & high   & low-medium \\
      \hline
      Memory complexity & medium        & medium  & low**  & medium*** \\
      \hline
      Time complexity  & medium         & low     & low**  & high*** \\
      \hline
      Robustness       & high           & high    & low    & high \\
      \hline
    \end{tabular}
  \end{center}
  \textit{\small{***As the algorithm has random component the worst case is very bad, but generally it should perform well}}\\
  \textit{\small{**The complexity is low mainly due to lack of features rather than having good complicity in $O$ notation.}}
  \caption{Comparison of different SLAM algorithms}
  \label{tab:path_algorithms}
\end{table} 


Whilst the feature based global path planning algorithm results seem appealing it's effectiveness is highly dependent on the actual map.
The worst case time and memory complexity is exponential which means it is not very robust and can halt in unfavourable conditions.
However some ideas from the algorithm can be used, namely how to pick nodes and neighbours for feature maps.
In the end we are leaning towards D* Lite as it should give a good performance and it doesn't seem too hard to implement.

\section{Obstacle Avoidance }

The environment where the robot functions is not dynamic. For obstacle avoidance we will plan relying on the path planning functionalities and on rerunning the path planning algorithm if there should be new features introduced or an obstacle detected. Depending on the chosen algorithm there are some obstacle avoiding features introduced already in them. 

Also we are panning to implement a simple threshold stop. If the robots LIDAR values decrease beyond the threshold and the robot is on a collision course, then the robot will stop, take needed actions and re-plan its route.  

There is an need for more complex algorithms if the environment is dynamic or if we are working with unknown environments. In that case different Obstacle avoidance algorithms can be used, like Artificial Potential Field Method \cite{Rostami19} or a simplified bubble band algorithm\cite{OA09} ,  \cite{OA15}. But for this project a simpler solution shall be used.

\section{Conclusion}

Both the RTT* and D* lite algorithms are applicable for our use case.
RTT* uses more computational capacity than the D*. Both of them are very robust and have some obstacle avoidance functionalities already in them. 
We decided to go with the D* lite algorithm because it can be implemented in a week and because the algorithm's time complexity is lower than RTT*'s. This would be more suitable for our case, where the robot has limited computational resources. 

\newpage
\begin{thebibliography}{9}

% example citation %
\bibitem{Karur21}
Karur, K.; Sharma, N.;
Dharmatti, C.; Siegel, J. A Survey of
Path Planning Algorithms for Mobile
Robots. Vehicles 2021, 3, 448–468.
https://doi.org/10.3390/
vehicles3030027

\bibitem[]{Ren22}
Ren, G., Liu, P., He, Z.: A
global path planning algorithm based on the feature
map. IET Cyber‐Syst. Robot. 4(1), 15–24 (2022).
https://doi.org/10.1049/csy2.12040

\bibitem{DstarWiki}
D*
https://en.wikipedia.org/wiki/D*

\bibitem[]{Rostami19}
Rostami, S.M.H., Sangaiah, A.K., Wang, J. et al. Obstacle avoidance of mobile robots using modified artificial potential field algorithm. J Wireless Com Network 2019, 70 (2019). https://doi.org/10.1186/s13638-019-1396-2

\bibitem{OA15}
Bhavesh, Vayeda Anshav. "Comparison of various obstacle avoidance algorithms." Int. J. Eng. Res. Technol 4 (2015): 629-632.

\bibitem{OA09}
Susnea, Ioan, Mînzu, Viorel, Vasiliu, G.. (2009). Simple, real-time obstacle avoidance algorithm for mobile robots. 

\bibitem{Karaman11}
Karaman, Sertac, and Emilio Frazzoli. "Sampling-based algorithms for optimal motion planning." The international journal of robotics research 30.7 (2011): 846-894.

\bibitem[]{RRT19}
Robotic Path Planning: RRT and RRT*, Tim Chinenov (14.02.2019) site: https://theclassytim.medium.com/robotic-path-planning-rrt-and-rrt-212319121378

\end{thebibliography}

\end{document}


